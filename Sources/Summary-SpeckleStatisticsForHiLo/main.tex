\documentclass{article}       %On doit définir la classe de document (
\usepackage[utf8]{inputenc}  %Pour écrire de base (doit tjrs être là)
\usepackage{supertabular} %Si jamais des tableaux vont sur plus qu'une page
\usepackage[T1]{fontenc}  %Pour faire des ë,ö,û,...
\usepackage{icomma}       %Pour écrire 1,8 au lieu de 1, 8 (espace après la virgule en mode math)

\usepackage{array}     %Pour pouvoir faire des matrices et tableaux
\usepackage{color}     %Pour écrire du texte en couleur
\usepackage{amsmath,mathtools}   %Pour écrire des équations
\usepackage{amssymb,amsfonts}   %Pour loader des variables, lettres, flèches, signes...
\usepackage{esint}     %Pour faire des intégrales fermées et doubles
\usepackage{multirow}  %Pour fusionner des cases dans un tableau
\usepackage{float}     %Pour les figures flottantes et placer les figures dans le texte
\usepackage{graphicx}  %Pour intégrer des graphiques, images et photos..
\usepackage{tikz}   %dessiner toute sorte de belles choses.. Pratiquement sans fin ;)
\usepackage[left=2.5cm,right=2.5cm,top=3cm,bottom=3cm]{geometry} %Pour controler les marges
\usepackage{booktabs}
\usepackage{hyperref}  %Pour mettre des liens de références dans le texte
\usepackage[francais]{babel}  %Le dictionnaire français et permet aussi de comprendre les lettres françaises
\usepackage{caption}   %Pour pouvoir mettre des légendes aux figures «captions»
\captionsetup[table]{name = Tableau}
\usepackage{subcaption}
\usepackage{gensymb} %Pour avoir des symboles
\usepackage{tabulary}  %Pour faire des tableaux
\usepackage{fancyhdr}  %Pour définir des en-têtes et pieds de page fancy 
\usepackage{siunitx}   %Pour écrire des unités du système international
\usepackage{textcomp} 
\usepackage[siunitx]{circuitikz}  %dessiner des circuits électroniques

\usepackage{afterpage}

\newcommand\blankpage{%
    \null
    \thispagestyle{empty}%
    \addtocounter{page}{-1}%
    \newpage}
    
\newcommand{\HRule}{\rule{\linewidth}{0.5mm}}

\usepackage{parskip}         %Pour gérer les alinéas et les espaces entre les paragraphes
\setlength{\parindent}{0pt}  %Alinéas
\setlength{\parskip}{15pt}  %Espace entre paragraphes

\setlength\extrarowheight{2pt}



\usepackage{fancyhdr}
\pagestyle{fancy}
\usepackage{lastpage}
\renewcommand\headrulewidth{1pt}
\fancyhead[L]{\textbf{Statistiques des speckles}}
\fancyhead[C]{}
\fancyhead[R]{Quentin Perry-Auger}
\renewcommand\footrulewidth{1pt}
\fancyfoot[C]{\textbf{Page \thepage}}
%\fancyfoot[R]{Jeudi $1^{er}$ Décembre 2016}


\newfloat{Graphique}{tbp}{lop}

\def\changemargin#1#2{\list{}{\rightmargin#2\leftmargin#1}\item[]}
\let\endchangemargin=\endlist 

\usepackage{listingsutf8}
\usepackage{listings}
\definecolor{codegreen}{rgb}{0,0.6,0}
\definecolor{codegray}{rgb}{0.5,0.5,0.5}
\definecolor{codepurple}{rgb}{0.58,0,0.82}
\definecolor{backcolour}{rgb}{0.95,0.95,0.92}

\lstset{inputencoding=utf8/latin1}
\lstdefinestyle{mystyle}{
    backgroundcolor=\color{backcolour},   
    commentstyle=\color{codegreen},
    keywordstyle=\color{magenta},
    numberstyle=\tiny\color{codegray},
    stringstyle=\color{codepurple},
    basicstyle=\footnotesize,
    breakatwhitespace=false,         
    breaklines=true,                 
    captionpos=b,                    
    keepspaces=true,                 
    numbers=left,                    
    numbersep=5pt,                  
    showspaces=false,                
    showstringspaces=false,
    showtabs=false,                  
    tabsize=2
}
\lstset{style=mystyle}

\usepackage[stable]{footmisc}
%\usepackage[bottom]{footmisc}


\begin{document}   %On commence TOUUUUUJOURS avec \begin{document} [....]  \end{document}

\begin{titlepage}  %Faire une page titre qui ne sera pas comptée, ni paginée
\begin{center}   %Pour centrer tout ce qu'il y a entre \begin et \end 
\includegraphics[scale=0.5]{fig/logo.png}\\
\vspace{1cm}


\HRule\\[0.4cm]
{\large\bfseries Statistiques des speckles}
\HRule\\[1.75cm]


\vfill

{\bfseries Quentin Perry-Auger (\url{quentin.perry-auger.1@ulaval.ca})}

\vfill

Stagiaire CERVO\\ Été 2018


\vfill

\end{center}

\textbf{\Large Résumé}

Cet ouvrage résume les différents aspects des statistiques des speckles. Les statistiques de premier ordre (en un seul point) sont d'abord étudiées pour l'amplitude, l'intensité et la phase. Les statistiques de deuxième ordre (en deux points) sont également abordées afin de trouver la taille approximative des speckles dans les différents axes de l'espace. Ensuite, l'intensité intégrée est abordée et le concept de speckles contraints qui s'applique aux fibres optiques est introduit et les caractéristiques des speckles observés à la sortie d'une fibre optique multimode sont détaillés. Enfin, une discussion sur les différents facteurs qui permettent de modifier le contraste d'un ensemble de speckles est présentée.

\end{titlepage}



\section{Somme de champs phaseurs}

Les speckles sont le résultat d'une somme de phaseurs de phase et/ou d'amplitude aléatoires. On peut donc écrire le champ électrique résultant en un point de l'ensemble de speckles de la façon suivante:

\begin{equation}
\label{Eq:SomPhas}
    A e^{j\theta} = \frac{1}{\sqrt{N}}\sum_{n=1}^{N}a_{n}e^{j\phi_{n}}
\end{equation}

où le facteur 1/$\sqrt{N}$ est introduit afin d'avoir un moment de second ordre fini lorsque le nombre de phaseurs $N$ de la somme tend vers l'infini. À partir de cette équation, on pose trois conditions sur les phaseurs sommés:\\
\begin{enumerate}
    \item Les amplitudes et les phases de deux phaseurs sont statistiquement indépendantes
    \item L'amplitude de tout phaseur est indépendante de sa phase et vice-versa
    \item Les phases sont distribuées uniformément sur l'intervalle [-$\pi$, $\pi$]
\end{enumerate}
\bigskip

Dans le cas où le nombre de phaseurs N tend vers l'infini (pour une discussion où ce n'est pas le cas, voir les sections 2.5 et 3.2.3 de \cite{Manuel}), après avoir utilisé le théorème central limite et effectué quelques manipulations mathématiques, on trouve que la fonction de densité de probabilités (FDP) de l'amplitude du champ résultant est de forme gaussienne:

\begin{equation}
\label{Eq:FDPAmp}
    P_{A}(A) = \frac{A}{\sigma^2}\exp\left(\frac{-A^2}{2\sigma^2}\right)
\end{equation}

De plus, il peut être pertinent de mentionner que d'additionner plusieurs sommes de phaseurs ne changent pas la forme l'équation \ref{Eq:SomPhas}.\\

\bigskip

\section{Statistiques de premier ordre}
\label{Sec:PremierOrdre}

À partir de l'équation \ref{Eq:FDPAmp} et de la relation entre l'intensité et l'amplitude ($I=A^2$), on peut obtenir la FDP de l'intensité en un point de l'ensemble de speckles,

\begin{equation}
\label{Eq:FDPInt}
    P_{I}(I) = \frac{1}{\overline{I}}\exp(-I/\overline{I}),
\end{equation}

et celle de la phase

\begin{equation}
\label{Eq:FDPPhase}
    P_{\theta}(\theta) = \frac{1}{2\pi}.
\end{equation}

Le résultat de l'équation \ref{Eq:FDPPhase} n'est pas surprenant puisque la distribution de la phase a été considérée comme uniforme (pour une discussion où ce n'est pas le cas, voir la section 2.6 de \cite{Manuel}). Les speckles ayant une FDP d'intensité de cette forme sont dits complètement développés (fully-developed speckles).\\

\bigskip


\section{Contraste et SNR}

Le contraste est défini comme étant la variance sur la moyenne d'une distribution. Dans le cas des speckles, il est donc donné par

\begin{equation*}
    C = \frac{\sigma_I}{\overline{I}}.
\end{equation*}

Ce concept est important pour la microscopie HiLo puisque c'est en calculant celui-ci sur la différence d'images ($I_{uniforme}-I_{speckle}$) que l'on peut extraire les basses fréquences spatiales contenues au plan focal. On introduit également le "signal to noise ratio" (SNR) qui est égal à l'inverse du contraste:

\begin{equation*}
    SNR = \frac{1}{C} = \frac{\overline{I}}{\sigma_I}.
\end{equation*}

Plusieurs facteurs peuvent influencer le contraste. Par exemple, lorsque N ensembles indépendants de speckles sont superposés, le contraste diminue jusqu'à un minimum de 1/$\sqrt{N}$ lorsque tous les ensembles ont une intensité moyenne égale. Ainsi, si on voulait augmenter le contraste des speckles à la sortie d'une fibre multimode, on pourrait y placer un polariseur pour ainsi passer de deux ensembles indépendants (un pour chaque composante de la polarisation) à un seul. Aussi, un ensemble de speckles complètement polarisé a un contraste plus élevé qu'un ensemble partiellement polarisé (pour un minimum de 1/$\sqrt{2}$ lorsque l'ensemble n'est pas du tout polarisé). Il est à noter que, dans ces deux cas, la FDP de l'intensité n'est plus une fonction exponentielle décroissante. Ces cas sont traités plus en détail au chapitre 3 de \cite{Manuel}. Les autres facteurs qui influencent le contraste sont détaillés à la section \ref{Sec:Reduction} de cet ouvrage.\\

\bigskip





\section{Statistiques de deuxième ordre}
\label{méthodes_expérimentales}

Il est possible de trouver les FDP jointes en deux points (dans l'espace ou dans le temps) d'un ensemble de speckles. Toutefois, celles-ci étant bien détaillées à la section 4.3 de \cite{Manuel}, elles ne seront pas énumérées ici. Nous nous intéressons plutôt aux dimensions des speckles dans l'espace, à commencer par celles dans le plan transverse. En assumant une propagation paraxiale et un ensemble de speckles formé par une surface non-uniforme à l'ordre de la longueur d'onde de la lumière (un diffuseur, par exemple), on obtient l'aire d'un speckle en intégrant le carré du coefficient de corrélation transverse complexe qui représente la fonction d'autocorrélation de l'amplitude normalisée (\cite{Manuel}, p. 75). Celui-ci est égal à:

\begin{equation}
\label{Eq:CoeffCorrel}
    \mu_{A}(\Delta x, \Delta y) = \frac{\Gamma_{A}(\Delta x, \Delta y)}{\Gamma_{A}(0,0)} = \frac{\int\int_{-\infty}^{\infty}I(\alpha, \beta)e^{-j\frac{2\pi}{\lambda z}(\alpha \Delta x + \beta \Delta y)}\text{d}\alpha\text{d}\beta}{\int\int_{-\infty}^{\infty}I(\alpha, \beta)\text{d}\alpha\text{d}\beta}
\end{equation}

où $I(\alpha, \beta)$ est l'illumination sur la surface diffusante et $x$ et $y$ sont les coordonnées sur le plan de speckles. Ce coefficient est obtenu à partir de l'équation de diffraction de Fresnel, donc l'équation \ref{Eq:CoeffCorrel} est valide pour le champ proche comme pour le champ lointain. Pour une surface de diffusion circulaire ou rectangulaire d'aire A, on obtient que l'aire d'un speckle est environ donnée par

\begin{equation*}
    A_{s} = \frac{(\lambda z)^2}{A}
\end{equation*}

où $\lambda$ est la longueur d'onde et $z$ est la distance axiale.

De façon plus générale, l'aire transverse d'un speckle est donnée par

\begin{equation*}
    A_{s} = \frac{(\lambda z)^{2}\int\int_{-\infty}^{\infty}I^{2}(\text{x,y})\text{dxdy}}{\left[\int\int_{-\infty}^{\infty}I(\text{x,y})\text{dxdy}\right]^2}
\end{equation*}

Pour ce qui est de la longueur des speckles dans la direction axiale (profondeur), celle-ci est définie dans \cite{Manuel} comme étant la largeur à mi-hauteur du coefficient de corrélation axial normalisé et est égal à $6,7\lambda(z/D)^2$ pour une surface circulaire de diamètre $D$ et à $4,8\lambda(z/L)^2$ pour une surface carrée de côté L. On remarque donc que la longueur d'un speckle dans la direction axiale est bien plus grande (elle varie en fonction de $z^2$) que les dimensions dans le plan transverse (où l'aire varie en fonction de $z^2$, donc les dimensions varient en fonction de $z$).\\

\bigskip

\section{Intensité intégrée}

Dans plusieurs applications, l'intensité n'est pas mesurée directement, mais est plutôt intégrée, que ce soit dans l'espace pour la microscopie HiLo ou dans le temps par un détecteur. Ainsi, on s'intéresse à la FDP de l'intensité une fois intégrée, que nous dénotons W. En approximant le profil d'intensité des speckles comme des boîtes rectangulaires, on obtient la forme approximative de la FDP (\cite{Manuel}, p.~112):

\begin{equation}
\label{Eq:IntIntUn}
    P(W) = \frac{(M/\overline{W})^{M}W^{M-1}\exp(-MW/\overline{W})}{\Gamma(M)}
\end{equation}

où $M$ est le nombre de degrés de liberté et est environ égale au rapport entre l'aire sur laquelle on intègre et l'aire approximative d'un speckle. La FDP de l'équation \ref{Eq:IntIntUn} est appelée 
distribution gamma. On remarque que lorsqu'on intègre sur seulement un speckle ($M\approx1$), la FDP devient une exponentielle décroissante comme pour un point d'un ensemble complètement développé.

\subsection{Dans le cas des fibres optiques - speckles contraints}
\label{Sec:SpeckCons}

Dans le cas d'une fibre optique, lorsque l'aire d'intégration $A_i$ est faible par rapport à l'aire totale du coeur de la fibre $A_c$ (ou encore le rapport entre le nombre de speckles N dans l'aire d'intégration et le nombre total de speckles N$_T$ sur une section du coeur de la fibre est faible), la distribution de l'intensité intégrée semble concorder avec la distribution gamma. Toutefois, les expériences montrent que les probabilités s'éloignent de ce type de distribution lorsque le rapport

\begin{equation*}
    \kappa = A_{i}/A_{c} = N/N_{T}
\end{equation*}

s'approche de l'unité. Une nouvelle formule plus représentative du comportement observé a été proposée par Goodman. Certaines suppositions sont toutefois nécessaires (\cite{Manuel}, p.~240):\\
\begin{enumerate}
    \item Tout le volume modal de la fibre est rempli (donc tous les modes possibles dans la fibre sont excités)
    
    \item Le nombre de modes se propageant dans la fibre est très grand
    
    \item La distribution des speckles dans le coeur de la fibre est spatialement stationnaire et aléatoire, ce qui est, d'après Goodman, une bonne approximation pour les fibres à saut d'indice, mais moins pour une fibre à gradient d'indice
    
    \item Le nombre de speckles est égal au nombre de modes se propageant dans la fibre, ce qui est une bonne approximation si tous les modes véhiculent la même puissance (à nouveau, ceci est une bonne approximation pour les fibres à saut d'indice, mais moins pour une fibre à gradient d'indice)
\end{enumerate}

\bigskip

Lorsque ces conditions sont remplies, la FDP en un point de la surface du coeur de la fibre a une forme exponentielle décroissante. La méthode proposée par Goodman pour trouver la nouvelle FDP est d'appliquer une contrainte à celle-ci, soit que la puissance totale $W_T$ est constante le long du coeur de la fibre. On cherche donc la FDP conditionnelle $P(W|W_T)$ qui est obtenue à partir de l'équation \ref{Eq:IntIntUn} et de la loi de Bayes:

\begin{equation*}
    P(W|W_T) = \frac{1}{W_T}\left(\frac{W}{W_T}\right)^{\kappa N_{T}-1}\left(1-\frac{W}{W_T}\right)^{(1-\kappa)N_{T}-1}\frac{\Gamma(N_{T})}{\Gamma(\kappa N_{T})\Gamma((1-\kappa)N_{T})}
\end{equation*}

\bigskip
\bigskip

\section{Speckles à la sortie d'une fibre multimode}

Dans une fibre multimode, la plus petite taille possible pour un speckle peut être trouvée en ne considérant que deux modes: un mode d'ordre très élevé et le mode d'ordre le plus bas \cite{FiberStats}. La taille minimale d'un speckle correspond alors à la taille d'une frange du patron d'interférence résultant à la surface de la fibre:

\begin{equation*}
    \Delta x = \lambda/(2\text{NA})
\end{equation*}

On trouves alors le nombre de speckles approximatif à la surface du coeur de rayon $a$ de la fibre:

\begin{equation*}
    N = \pi\left(\frac{2a\text{NA}}{\lambda}\right)^2
\end{equation*}

On remarque donc que l'approximation mentionnée à la section \ref{Sec:SpeckCons} qui stipule que le nombre de speckles est égal au nombre de modes est assez bonne pour une fibre a saut d'indice considérant qu'une telle fibre contient environ $2(\pi a\text{NA}/\lambda)^2$ modes. On en conclut également qu'une fibre à saut d'indice aura des speckles plus fins et en plus grand nombre qu'une fibre à gradient d'indice qui contient deux fois moins de modes.


\subsection{Speckles dans le champ proche (Fresnel)}

À la sortie immédiate de la fibre, si on intègre l'intensité sur une faible portion du coeur, puisque les speckles résultent d'un grand nombre de modes dont les phases sont aléatoires, la FDP devrait être près d'une exponentielle décroissante. Ceci semble également concorder avec les résultats obtenus lors de la mesure du signal to noise ratio (SNR) dans \cite{Constrained}. Théoriquement, puisque la distribution est la même que celle obtenue par un diffuseur et que c'est la méthode utilisée par la plupart des microscopes HiLo, en imageant le bout de la fibre sur l'échantillon, l'imagerie serait possible. Toutefois, dans ce cas, le laser ne serait pas collimé et la taille des speckles changerait avec la distance et l'illumination ne serait pas uniforme (au niveau de la symétrie générale, évidemment, puisqu'une illumination avec speckles n'est pas uniforme).


\subsection{Speckles dans le champ lointain (Fraunhoffer)}

Dans le champ lointain, la distribution spatiale de l'ensemble de speckles cesse de varier avec la distance (\cite{Commerce}, p.~1). On obtient donc une illumination symétrique. Pour ce qui est des statistiques de cette distribution, celles-ci sont peu documentées explicitement. Toutefois, on remarque que l'ensemble de speckles au champ lointain ressemble énormément à celui au champ proche (\cite{FiberStats}, Fig.~1~a). Les résultats de la FDP de l'intensité en régime de Fraunhoffer pour une courte fibre optique à saut d'indice semble également se rapprocher de la distribution exponentielle négative (\cite{FiberStats}, Fig.~5~a) et celle-ci s'éloigne peu de cette forme même en augmentant la longueur de la fibre (la FDP s'applatit toutefois légèrement). Ceci semble avoir du sens, puisque la distribution spatiale du champ électrique à la sortie de la fibre est aléatoire si le nombre de modes est élevé, donc le champ lointain, qui est la somme de toutes ces contributions aléatoires, devrait également l'être.\\

\bigskip

\section{Réduction de speckles (paramètres qui affectent le contraste)}
\label{Sec:Reduction}

Contrôler le contraste du patron de speckles est important en microscopie HiLo, puisqu'un meilleur contraste permet d'avoir des mesures plus précises lors de l'application de l'algorithme HiLo. Tel que mentionné plus tôt, des paramètres comme le nombre d'ensembles de speckles indépendants ou encore la polarisation affectent le contraste. Toutefois, plusieurs autres paramètres influencent celui-ci, dont certains sont propres aux fibres optiques.

\subsection{Moyennage temporel}

Si le patron de speckles fluctue dans le temps à une fréquence très élevée en comparaison avec le temps d'intégration de la caméra, la distribution spatiale est moyennée et on obtient une illumination uniforme. Ceci peut être effectué en faisant vibrer la fibre optique assez rapidement. Il est également possible d'augmenter cet effet en ajoutant un diffuseur à la sortie de la fibre au prix d'une perte de puissance envoyée sur l'échantillon puisqu'il y a alors plus de divergence (\cite{Manuel}, section 5.2).

\subsection{Longueur de la fibre}

Tout d'abord, en allongeant la fibre, les modes d'ordre supérieur interfèrent de plus en plus avec les modes d'ordre inférieur, ce qui tend à diminuer le contraste des speckles (\cite{FiberStats}, Fig.~6). Aussi, si la longueur de la fibre est assez importante, les modes peuvent cesser d'interférer puisque la différence de chemin optique entre ceux-ci peut dépasser la longueur de cohérence de la source. La longueur critique où les rayons parallèles à l'axe optique et les rayons incidents à l'angle critique cessent d'interférer dans une fibre à saut d'indice est \cite{Coherence}

\begin{equation*}
    L_{c} = \frac{c n_{2} t_{c}}{n_{1}(n_{1} - n_{2})}
\end{equation*}

où $t_c$ est le temps de cohérence de la source.

\subsection{Largeur du spectre de la source}

Puisque la largeur du spectre de la source est inversement proportionnelle à son temps de cohérence \cite{Coherence}, augmenter la largeur du spectre diminue rapidement la longueur critique à laquelle les modes cessent d'interférer.

\subsection{Courbures dans la fibre optique}

Une fibre entortillée a un contraste plus faible qu'une fibre droite. Le contraste diminue jusqu'à atteindre un plateau où une redistribution stable des modes est atteinte (\cite{FiberStats}, Fig.~8).

\subsection{Diversité angulaire}

Une réduction du contraste peut également être obtenue avec de la diversité angulaire. Dans \cite{Angular}, ceci est effectué en envoyant plusieurs faisceaux à différents angles d'incidence sur un diffuseur à l'entrée de la fibre.



\newpage

\section*{Annexe I - Termes et concepts de probabilités et statistiques}
\label{AnnexeI}

NB: Dans cette annexe, la notation $E[X]$ est utilisée pour dénoter la moyenne plutôt que $\overline{X}$.

\begin{itemize}
    \item \textbf{Espérance (expected value, mean)}: valeur moyenne qu'on s'attend à retrouver si on répète l'expérience un grand nombre de fois.
    
    \begin{equation*}
        E[X] = \int_{-\infty}^{\infty}X P(x) dx
    \end{equation*}
    P(x): densité de probabilité\\
    
    
    \item \textbf{Écart-type (standard deviation)}: mesure de l'écart entre les valeurs et la moyenne (mesure de la dispersion d'un échantillon).
    
    \begin{equation*}
        \sigma_X = \left(E[(X - E[X])^2]\right)^{1/2} = \left(E[X^2]-E[X]^2\right)^{1/2}
    \end{equation*}
    
    \bigskip
    
    
    \item\textbf{Variance}: carré de l'écart-type, donc mesure de la dispersion également.\\
    
    \item\textbf{Covariance}: variance conjointe entre deux variables. Elle prend des grandes valeurs si les écarts des variables correspondent.
    
    \begin{equation*}
        cov(X,Y) = E[(X-E[X])(Y-E[Y])]
    \end{equation*}
    
    \bigskip
    
    \item\textbf{Corrélation}: Indication du degré de dépendance entre deux variables aléatoires.
    
    \begin{equation*}
        corr(X,Y) = \frac{cov(X,Y)}{\sigma_{X}\sigma_{Y}}
    \end{equation*}
    
    \bigskip
    
    \item\textbf{Autocorrélation}: corrélation d'une variable avec elle-même (décalée d'un certain temps ou d'une certaine valeur). Ceci permet de trouver par exemples des segments périodiques dans un signal bruité.\\
    
    \item\textbf{Fonction caractéristique}: Pour une variable X, la fonction caractéristique est la valeur attendue (expected value, mean) de exp(j$\omega$x), ou encore, la transformée de Fourier inverse de la densité de probabilité. En 2D, la fonction caractéristique est la transformée de Fourier inverse en 2D de la densité de probabilité jointe.
    
    
    
\end{itemize}
\newpage


\begin{thebibliography}{9}

\bibitem{Manuel} Goodman, Joseph W., \emph{Speckle phenomena in optics: theory and applications}, Roberts and Company Publishers (2007).

\bibitem{Constrained} Goodman, J. W., Rawson, E. G., \emph{Statistics of modal noise in fibers: a case of constrained speckle}, Optics Letter Vol. 6 No. 7 (1981).

\bibitem{Commerce} Kim, E. M, Franzen, D. L., \emph{Measurements of Far-Field and Nead-Field Radiation Patterns from Optical Fiber}, US National Bureau of Standards (1981).

\bibitem{FiberStats} Masaaki, Imai, \emph{Statistical Properties of Optical Fiber Speckles}, Bulletin of the Faculty of Engineering, Hokkaido University (1985).

\bibitem{Coherence} Crosignani, B. et al., \emph{Speckle-pattern visibility of light transmitted through a multimode optical fiber}, Optical Society of America Vol. 66 No. 11 (1976).

\bibitem{Angular} Mehta, D. S. et al., \emph{Laser speckle reduction by multimode optical
fiber bundle with combined temporal,
spatial, and angular diversity},  Applied Optics Vol. 51 No. 12 (1976).

\end{thebibliography}

\end{document}